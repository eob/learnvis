\documentclass{sigchi}

% Load basic packages
\usepackage{balance}  % to better equalize the last page
\usepackage{graphics} % for EPS, load graphicx instead
\usepackage{times}    % comment if you want LaTeX's default font
\usepackage{url}      % llt: nicely formatted URLs
\usepackage{color}
\usepackage{listings}
\usepackage{wrapfig}
\usepackage{graphicx}
\usepackage{caption}
\usepackage{subcaption}
\usepackage[english]{babel}
\usepackage{booktabs}
\usepackage{url}      % llt: nicely formatted URLs
\usepackage{graphicx}

\include{hacks}

% llt: Define a global style for URLs, rather that the default one
\makeatletter
\def\url@leostyle{%
  \@ifundefined{selectfont}{\def\UrlFont{\sf}}{\def\UrlFont{\small\bf\ttfamily}}}
\makeatother
\urlstyle{leo}

% To make various LaTeX processors do the right thing with page size.
\def\pprw{8.5in}
\def\pprh{11in}
\special{papersize=\pprw,\pprh}
\setlength{\paperwidth}{\pprw}
\setlength{\paperheight}{\pprh}
\setlength{\pdfpagewidth}{\pprw}
\setlength{\pdfpageheight}{\pprh}

% Make sure hyperref comes last of your loaded packages, 
% to give it a fighting chance of not being over-written, 
% since its job is to redefine many LaTeX commands.
\usepackage[pdftex]{hyperref}
\hypersetup{
pdftitle={SIGCHI Conference Proceedings Format},
pdfauthor={LaTeX},
pdfkeywords={SIGCHI, proceedings, archival format},
bookmarksnumbered,
pdfstartview={FitH},
colorlinks,
citecolor=black,
filecolor=black,
linkcolor=black,
urlcolor=black,
breaklinks=true,
}

% create a shortcut to typeset table headings
\newcommand\tabhead[1]{\small\textbf{#1}}

% End of preamble. Here it comes the document.
\begin{document}

% for 
\title{LEARN ALL THE VIS!}

% CHI 2014 Full Paper Submission #455

\numberofauthors{4}
\author{
  \alignauthor 1st Author Name\\
    \affaddr{Affiliation}\\
    \affaddr{Address}\\
    \email{e-mail address}\\
    \affaddr{Optional phone number}
  \alignauthor 2nd Author Name\\
    \affaddr{Affiliation}\\
    \affaddr{Address}\\
    \email{e-mail address}\\
    \affaddr{Optional phone number}    
  \alignauthor 3rd Author Name\\
    \affaddr{Affiliation}\\
    \affaddr{Address}\\
    \email{e-mail address}\\
    \affaddr{Optional phone number}    
  \alignauthor 4th Author Name\\
    \affaddr{Affiliation}\\
    \affaddr{Address}\\
    \email{e-mail address}\\
    \affaddr{Optional phone number}    
}

% Teaser figure can go here
%\teaser{
%  \centering
%  \includegraphics{Figure1}
%  \caption{Teaser Image}
%  \label{fig:teaser}
%}

\maketitle

\begin{abstract} 
 
Why make something when we can make it for you?

\end{abstract}

\section{Introduction}

This is the intro




% Balancing columns in a ref list is a bit of a pain because you
% either use a hack like flushend or balance, or manually insert
% a column break.  http://www.tex.ac.uk/cgi-bin/texfaq2html?label=balance
% multicols doesn't work because we're already in two-column mode,
% and flushend isn't awesome, so I choose balance.  See this
% for more info: http://cs.brown.edu/system/software/latex/doc/balance.pdf
%
% Note that in a perfect world balance wants to be in the first
% column of the last page.
%
% If balance doesn't work for you, you can remove that and
% hard-code a column break into the bbl file right before you
% submit:
%
% http://stackoverflow.com/questions/2149854/how-to-manually-equalize-columns-
% in-an-ieee-paper-if-using-bibtex
%
% Or, just remove \balance and give up on balancing the last page.
%
\balance
\footnotesize
% If you want to use smaller typesetting for the reference list,
% uncomment the following line:
% \small
\bibliographystyle{acm-sigchi}
\bibliography{main}

\end{document}
